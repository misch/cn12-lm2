\section{Encapsulation}

\subsection{Link Layer VPNs (Layer 2)}
Beispiele f�r VPNS auf dieser Ebene sind Integrated Services Digital Network (ISDN), Frame Realy und Asynchronous Transfer Mode. Auch Virtual Local Networks funktionieren auf �hnliche Weise auf dieser Ebene.

\paragraph{Vorteile}
\begin{itemize}
\item Verbindungsorientiert, Pakete m�ssen nicht geroutet werden sondern werden �ber einmal erstellten Link �bertragen.
\item QoS auf dieser Ebene schon garantiert, muss nicht noch zus�tzlich implementiert werden.
\end{itemize}

\paragraph{Nachteile}
\begin{itemize}
\item Braucht homogene Netzwerkstruktur
\item IP-Ebene muss trotzdem noch verwaltet werden, Mehraufwand auf zwei Ebenen!
\end{itemize}

\subsection{Network Layer VPNs (Layer 3)}
IP Pakete werden als Payload in ein anderes IP Paket verpackt(IP in IP, {\em IPIP}), dann versendet. Dass innere Paket kann dabei verschl�sselt werden und garantiert so Sicherheit, das �ussere jedoch nicht. Der VPN Server dient dann als Interface, der den �usseren Header entfernt & das innere Paket entschl�sselt. 

Firewalls k�nnen eingesetzt werden um verschl�sselte Verbindungen zu erzwingen. 