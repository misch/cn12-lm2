\section[SA \& SPD]{Security Association (SA) and Security Policy Database (SPD)}

Sowohl AH als auch ESP m�ssen an einem gewissen Punkt im Netzwerk eine Ver�nderung an IP-Paketen vornehmen. Die involvierten IPSec-Knoten bilden Sender-Empf�nger-Paare; der Sender ver�ndert ein IP-Paket, der Empf�nger macht die Ver�nderung r�ckg�ngig. Die Beziehung zwischen Sender um Empf�nger wir durch eine Security Association (SA) beschrieben.\newline Eine SA beschreibt also nur genau eine Transformation und die zugeh�rige R�cktransformation. Beim Einsatz mehrerer Sicherheitsdienste m�ssen also auch mehrere SA's aufgebaut werden.\newline Unter IPSec werden durch die SA verschiedene Sachen spezifiziert:

Eine SA wird durch einen Security Parameter Index (SPI, 32 bits), die Ziel-Adresse und einen Security Protocol Identifier (zur Bezeichnung des �bertragungsverfahrens; AH/ESP) festgelegt.\newline Der Sender schreibt den SPI in das entsprechende Feld der IP-Protokoll-Erweiterung und der Empf�nger kann dann mit Hilfe dieser Information die richtige Security Association identifizieren und kann so die zuvor vorgenommene Transformation r�ckg�ngig machen und auf das urspr�ngliche Paket zugreifen.\newline Jede kommunizierende Einheit kann in beliebig viele Sicherheitsverbindungen (SA's) involviert sein. Die Spezifikationen der Sicherheitsverfahren sind lokal in jedem IPSec-Knoten lokal in der Security Policy Database (SPD) gespeichert. Die SPD enth�lt verschiedene Eintr�ge f�r ein- und ausgehenden Datenverkehr. Durch die SPD wird bestimmt, ob ein Datenstrom verschl�sselt werden muss oder im Klartext �bertragen werden kann. Wenn verschl�sselte Daten versendet werden, muss die SPD eine Referenz auf die entsprechende SA haben.