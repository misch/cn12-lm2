\section{Transport and Tunnel Mode}
Sowohl bei ESP als auch bei AH gibt es zwei Modi: transport mode und tunnel mode.
\subsection{Transport Mode}
Beim transport mode werden die Nutzdaten und ein Teil des IP-Headers verschl�sselt. Der IP-Header wird durch das Hinzuf�gen von neuen Feldern erweitert.\newline
Dieser Modus erm�glicht die Verwendung von IPSec �ber eine End-zu-End-Verbindung.
\subsection{Tunnel Mode}
Im tunnel mode wird dem Paket ein komplett neuer IP-Header angef�gt. Der Modus ist ideal zum Implementieren eines VPN-Tunnels. Sowohl AH als auch ESP k�nnen f�r einen IP-VPN-Tunnel verwendet werden. Tunneling packt das urspr�ngliche IP-Paket ein (ESP) und f�gt einen neuen IP-Header hinzu, der die Adresse des IPSec-Gateways enth�lt. Dieser Modus erm�glichet es, nicht-routbare IP-Adressen (oder andere Protokolle) �ber ein �ffentliches Netzwerk �bertragen, da die Adressen im inneren Header versteckt sind. Auch die urspr�ngliche Netzwerk-Topologie wird versteckt (Privacy).